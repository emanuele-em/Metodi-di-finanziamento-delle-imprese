\chapter{Finanziamento delle imprese in mercati imperfetti}
\label{sec:Finanziamento delle imprese in mercati imperfetti}

\begin{itemize}
    \item Allineamenti diversi tra proprietario, manager e investori, moral hazard e decisioni non contrattabili
    \item Asimmetrie informative
\end{itemize}

\section{Review Modigliani e Miller}

Secondo il modello semplificato di Modigliani Miller ci sono le seguenti assunzioni:

\begin{itemize}
    \item Perfetta competizione
        \subitem Non ci sono costi di transazione, le imprese e gli individui non pagano tasse
        \subitem Tutti gli agenti (imprese e individui) sono price taker, possono quindi tutti contrarre dei prestiti nelle medesime condizioni
        \subitem L'informazione è completa
    \item Non ci sono opportunitá di arbitraggio: un'opportunitá di arbitraggio è la Possibilitá di ottenere profitti \textbf{certi} a fronte di \textbf{rischi futuri nulli}, le opportunitá di arbitraggio, data la loro potenza si esauriscono in tempi brevissimi, tutti gli arbitraggisti sono alla ricerca di queste opportunitá e vengono sfruttati algoritmi per poter essere competivi. Le opportunitá di arbitraggio garantiscono un continuo aggiustamento del prezzo.
    \item Le decisioni aziendali non hanno influenza sui cash-flow generati dagli investimenti
\end{itemize}

\subsection{Prima Proposizione di Modigliani Miller}
\begin{theorem}
    Il valore di mercato delle imprese è indipendente dalla sua struttura di capitale
\end{theorem}

Secondo una differente lettura è possibile affermare che tramite il \textit{market value balance sheet of the firm}, ovvero un bilancio non contabile ma basato solo sui valori di mercato presenti, possiamo capire che in un mondo ideale di Modigliani Miller dovremmo preoccuparci solamente dell'attivo e non del passivo: tutte le variazioni di passività sono quindi un semplice riflesso delle variazioni delle attività che diventano quindi un loro sottostante.

Questa affermazione è di fondamentale importanza perchè ci permette di comprendere quale sia l'importo massimo di finanziamento ottenibile: \textbf{Il valore dell'attivo}.

I.e. Se un impresa in possesso di uno stabilimento funzionante ha intenzione di finanziare un nuovo progetto, può ottenere al massimo il valore dello stabilimento come finanziamento, i flussi di cassa attesi finanzieranno solo al massimo il valore di mercato dell'attivo.

In altre parole:
Un'impresa può \textit{pledge} (impegnare) cioè può credibilmente offrire agli investitori esterni, al massimo il valore presente degli assets in suo possesso. È quindi presente un limite naturale al finanziamento esterno.

\subsection{Seconda proposizione di Modigliani Miller}
\begin{theorem}
    \begin{equation} \label{eq:proposition2}
        \begin{gathered}
            r_E=r_U+\frac{D}{E}*(r_U-r_D)
        \end{gathered}
    \end{equation}
\end{theorem}