\chapter{Finanziamento delle imprese in mercati imperfetti}

\label{sec:Finanziamento delle imprese in mercati imperfetti}

\begin{itemize}
    \item Allineamenti diversi tra proprietario, manager e investori, moral hazard e decisioni non contrattabili
    \item Asimmetrie informative
\end{itemize}

\section{Review Modigliani e Miller}

Secondo il modello semplificato di Modigliani Miller ci sono le seguenti assunzioni:

\begin{itemize}
    \item Perfetta competizione
          \subitem Non ci sono costi di transazione, le imprese e gli individui non pagano tasse
          \subitem Tutti gli agenti (imprese e individui) sono price taker, possono quindi tutti contrarre dei prestiti nelle medesime condizioni
          \subitem L'informazione è completa
    \item Non ci sono opportunitá di arbitraggio: un'opportunitá di arbitraggio è la Possibilitá di ottenere profitti \textbf{certi} a fronte di \textbf{rischi futuri nulli}, le opportunitá di arbitraggio, data la loro potenza si esauriscono in tempi brevissimi, tutti gli arbitraggisti sono alla ricerca di queste opportunitá e vengono sfruttati algoritmi per poter essere competivi. Le opportunitá di arbitraggio garantiscono un continuo aggiustamento del prezzo.
    \item Le decisioni aziendali non hanno influenza sui cash-flow generati dagli investimenti
\end{itemize}

\subsection{Prima Proposizione di Modigliani Miller}
\begin{theorem}
    Il valore di mercato delle imprese è indipendente dalla sua struttura di capitale
\end{theorem}

Secondo una differente lettura è possibile affermare che tramite il \textit{market value balance sheet of the firm}, ovvero un bilancio non contabile ma basato solo sui valori di mercato presenti, possiamo capire che in un mondo ideale di Modigliani Miller dovremmo preoccuparci solamente dell'attivo e non del passivo: tutte le variazioni di passività sono quindi un semplice riflesso delle variazioni delle attività che diventano quindi un loro sottostante.

Questa affermazione è di fondamentale importanza perchè ci permette di comprendere quale sia l'importo massimo di finanziamento ottenibile: il valore delle attività.

I.e. Se un impresa in possesso di uno stabilimento funzionante ha intenzione di finanziare un nuovo progetto, può ottenere al massimo il valore dello stabilimento come finanziamento, i flussi di cassa attesi finanzieranno solo al massimo il valore di mercato dell'attivo.

In altre parole:
Un'impresa può \textit{pledge} (impegnare) cioè può credibilmente offrire agli investitori esterni, al massimo il valore presente degli assets in suo possesso. È quindi presente un limite naturale al finanziamento esterno.

\subsection{Seconda proposizione di Modigliani Miller}

\begin{equation} \label{eq:proposition2}
    \begin{gathered}
        r_E=r_U+\frac{D}{E}*(r_U-r_D)
    \end{gathered}
\end{equation}
\begin{conditions*}
    r_E &   ritorno atteso nelle imprese levered (costo del capitale Equity)\\
    r_U         &   ritorno atteso imprese Unlevered \footnote{Le imprese Unlevered sono le imprese che non fanno ricorso al capitale di debito, chiamate imprese all equity} \\
    r_D         &   ritorno atteso sul debito (costo del debito o interessi sul debito) \\
    \frac{D}{E} &   tasso di leva, leverage ratio sulle imprese levered \footnote{Le imprese Levered sono le imprese che fanno ricorso al capitale di debito}
\end{conditions*}

\begin{figure}[H]
    \centering
    \includegraphics[width=.7\linewidth]{images/chapter1/1.jpg}
    \caption{Grafico condizioni base Modigliani Miller}
    \label{fig:curva_mm}
\end{figure}

\begin{itemize}
    \item Il costo dell'equity (\(r_E\)) aumenta perchè l'equity è rischioso
    \item Senza rischi di bancarotta (ipotesi base), il costo del debito (\(r_D\)) è costante, pari al tasso di interesse
    \item Il costo del capitale (\(r_{WACC}\)) è indipendente dalla struttura del capitale, valore di ritorno atteso delle imprese Unlevered (\(r_U\))
\end{itemize}

Il tasso di interesse quindi è più alto tanto maggiore è la rischiositá.

\section{Moral Hazard} \label{sec:moralHazard}

Quando degli individui chiedono un finanziamento, il creditore assegna un credit score, il credit score viene formulato in base a dei criteri che dipendono dal creditore.

Il credit score riassume la capacità di rimborsare i debiti. Tanto più è alto il tasso di interesse tanto più sarà rischioso il progetto per il quale si richiede il credito, tanto più sará basso il credit score assegnato.

I finanziamenti ad alto tasso di interesse invogliano i creditori (imprese) ad effettuare investimenti ad alto rischio, perchè gli investimenti ad alto rischio sono anche quelli più remunerativi e permettono, in caso di successo, di riuscire a guadagnare oltre che rimborsare gli interessi molto alti. Questi incentivi ci sono solamente se le imprese sono a responsabilitá limitata per gli imprenditori, ovvero se gli imprenditori non rispondono con il capitale proprio neanche nel caso di insolvenza.

Si ha \textit{Moral hazard} quando chi prende decisioni non ha gli interessi allineati con chi possiede l'impresa (azionisti), cioè quando la proprietá è separata dal controllo.

\subsubsection{Modello base di Moral Hazard}

\begin{itemize}
    \item Un imprenditore che decide, ha bisogno di finanziamenti e ha due progetti differenti da realizzare in via esclusiva (o il progetto \(H\) o il progetto \(L\))
    \item I due progetti hanno bisogno dello stesso finanziamento \(I\) per poter essere realizzati
    \item L'imprenditore è in possesso di liquiditá ma non sufficiente a coprire \(I\)
    \item Il progetto \(H\) a \(t=1\)
          \subitem paga \(R>0\)  se ha successo con probabilitá \(p_H\)
          \subitem paga \(0\) se fallisce con probabilitá  \(1-p_H\)
    \item Il progetto \(L\) a \(t=1\)
          \subitem paga \(R>0\)  se ha successo con probabilitá \(p_L\)
          \subitem paga \(0\) se fallisce con probabilitá  \(1-p_L\)
    \item \(\Delta p=p_H-p_L>0\) questo significa che \(p_H > p_L\)
    \item l'imprenditore preferisce, per questioni personali, il progetto \(L\), ha quindi un beneficio privato \(B>0\)
    \item \(H\) è efficiente mentre \(L\) no: \(p_H*R-I > 0\), \(p_L*R-I < 0\)
    \item Tutti gli investitori sono neutrali al rischio\footnote{Un operatore economico si dice neutrale al rischio quando le sue preferenze lo rendono indifferente al compiere un'azione il cui risultato dipende da un elemento aleatorio, oppure rimanere nella situazione in cui si trova.}, il tasso di interesse risk free è uguale a \(0\)
    \item Gli investitori sono competitivi, all'equilibrio hanno un profitto nullo
\end{itemize}

In questo caso il Moral Hazard esiste in quanto gli investitori non possono forzare la scelta di uno o l'altro progetto. Quando viene firmato il contratto di finanziamento l'imprenditore sceglie il progetto, se il progetto scelto fallisce è impossibile provare il Moral Hazard anche perchè questo può essere addirittura inconsapevole. Escludendo le frodi l'ipotesi alla base del modello non è quindi realistica.

Se il progetto ha successo l'imprenditore guadagna \(R_B\), gli investitori guadagnano \(R-R_B=R_L\) dove 
\begin{itemize}
    \item \(R_B=R_{borrower}=\text{Compenso dell'imprenditore}\)
    \item \(R_L=R_{lender}=\text{Compenso degli investitori esterni}\)
\end{itemize}
Se il progetto il payoff è nullo.

Per compensare il Moral Hazard occorre incentivare l'imprenditore a scegliere \(H\), dato che preferisce \(L\), portando così in equilibrio la preferenza.

\subsection{Vincolo di compatibilità degli incentivi}

\begin{equation}\label{eq:compatibilita}
    \begin{gathered}
        p_H*R_B \geq p_l*R_B + B \\
        R_B \geq \frac{B}{\Delta p}
    \end{gathered}
\end{equation}
\begin{conditions*}
    p_H*R_B &   valore atteso del guadagno dell'imprenditore \\
    p_l*R_B + B &   ammontare dei benefici privati dell'imprenditore se sceglie il suo progetto preferito e il progetto ha successo. Il beneficio è quindi tradotto in €, si calcola sapendo quando sarebbe disposto a pagare l'imprenditore per scegliere il progetto \(L\) piuttosto che il progetto \(H\)
\end{conditions*}

All'interno dell'equazione \ref{eq:compatibilita} andrebbe considerato anche il caso di insuccesso con probabilitá \(1-p_H\) e \(1-p_L\) ma in questo specifico caso di esempio il beneficio in caso di insuccesso è pari a \(0\) perciò non viene considerato

L'incentivo genera un limite che i finanziatori esteri si aspettano giá, offrendo infatti un \(R_B\) maggiore all'imprenditore gli investitori sono consapevoli che a loro spetterá solamente il residuo corrispondente al cosiddetto \textit{Pledgeble Income}

\begin{equation}\label{eq:pledgebleIncomeStandard}
    p_H*(R-\frac{B}{\Delta p})
\end{equation}
Infatti il valore \(\frac{B}{\Delta p}\) è il compenso dell'imprenditore. Questo compenso viene dato solo in caso di successo del progetto, in caso di insuccesso infatti tutti gli individui ottengono un payoff nullo.

\subsection{Vincolo di Partecipazione}

Il \textit{vincolo di partecipazione} per gli investitori corrisponde al limite per gli investitori esterni sotto il quale non sono disposti a concedere il finanziamento:

\begin{equation} \label{eq:vincoloPartecipazione}
    p_H*(R-\frac{B}{\Delta p}) \geq I - A
\end{equation}
\begin{conditions*}
    p_H*(R-\frac{B}{\Delta p})  &   Pledgeble Income \\
    I-A &   Ammontare dei finanziamenti esterni, in particolare \(I\) è l'ammontare dell'investimento totale mentre \(A\) è l'investimento personale dell'imprenditore
\end{conditions*}

Isolando \(A\) dalla \ref{eq:vincoloPartecipazione}:

\begin{equation}
    A \geq p_H*\frac{B}{\Delta p} - (p_H*R - I)
\end{equation}
\begin{conditions*}\label{eq:vincoloPartecipazioneA}
    p_H*\frac{B}{\Delta p}  &   \(p_H*R_B\) = Payoff atteso \\
    p_H*R-I &   NPV del progetto (Net Present Value: valore attuale netto)
\end{conditions*}
Ne consegue che \(A\), nonchè l'investimento personale dell'imprenditore, deve avere un valore minimo, il progetto quindi non riceve finanziamenti esterni se prima non ha ricevuto finanziamenti direttamente dall'imprenditore.

La conseguenza è duplice;
\begin{itemize}
    \item \textbf{Cattive notizie per l'imprenditore}: se non si ha la disponibilitá di un capitale di almeno \(A\), il progetto non può essere eseguito.
     
    \item \textbf{Buone notizie per l'imprenditore}: se l'imprenditore disponde di un capitale investibile \(A\) allora, una volta che il progetto ha successo, l'imprenditore stesso incassa una quasi rendita, grazie al vincolo di compatibilitá degli incentivi rappresentato dall'equazione \ref{eq:compatibilita} riesce ad appropriarsi di tutto il valore attuale netto del progetto (NVP).
    
    \[
        p_H*R_B-A = \underbrace{p_H*R-I}_{all'imprenditore} > 0
    \]
          
    La difficoltá sta quindi nel superare il vincolo di partecipazione \ref{eq:vincoloPartecipazioneA} dovuto al Moral Hazard, superato il vincolo l'imprenditore è in grado di guadagnare molto di più degli investitori esterni.
\end{itemize}

\subsubsection{Domande di riepilogo}

\begin{enumerate}
    \item Nel modello presentato, gli investitori si aspettano un interesse positivo in caso in cui l'imprenditore scelga il progetto \(H\)?

    Supponendo che il vincolo di partecipazione \ref{eq:vincoloPartecipazioneA} sia soddisfatto il tasso di interesse rimane comunque positivo, il rischio del progetto non è infatti pari a zero, unica condizione che permetterebbe un tasso idealmente nullo.
    \begin{equation}\label{eq:condizioner}
        \begin{split}
            p_H(1+r)    &=  1\\
            r   &=  \frac{1}{p_H}-1 \geq 0 
        \end{split}
    \end{equation}
    L'equzione \ref{eq:condizioner} è valida perchè essendo \(p_H\) una probabilitá non potrà avere un valore superiore a 1 (caso ideale) in nessun caso.

    È importante capire che i finanziatori fissano un tasso di interesse \(r\) a priori, da quel tasso di interesse poi si calcola a ritroso il valore di \(R_B\) e \(R_L\), il guadagno degli investitori è quindi implicito nel tasso di interesse stabilito.

    \item Quanto incide sulla possibilitá di ottenre un finanziamento se gli investitori ottengono una quota del payoff del progetto molto più bassa rispetto all'imprenditore?

    Dato il vincolo di compatibilità degli incentivi \ref{eq:compatibilita} se \(R_B\) aumenta causa una diminuzione di \(R_L\).

    Dato il vincolo \ref{eq:vincoloPartecipazioneA}, aumentando il termine \(\frac{B}{\Delta p}\) il vincolo diventa piu stringente perché per rispettarlo occorre che \(A\) aumenti in proporzione.

    La conseguenza è che quando conferisco una maggiore quota del payoff all'imprenditore piuttosto che agli investitori è più difficile ottenere il finanziamento ma sará più semplice verificare il vincolo di compatibilità degli incentivi \ref{eq:compatibilita} (\(R_B\geq \frac{B}{\Delta p}\))
\end{enumerate}

\subsection{Soluzioni al Moral Hazard}

Se l'imprenditore ha abbastanza soldi financia il so progetto, se non li ha allora non lo finanzia: questa ipotesi è troppo semplicistica, ci sono casi in cui A=0 e il progetto viene comunque finanziato. Escludendo i sussidi perpetrati dalle istituzioni, che rendono il finanziatore in perdita, ci sono diverse alternative per sconfiggere almeno in parte il Moral Hazard che rendono la soluzione comunque efficace per il finanziatore. 

\begin{itemize}
    \item Micro credito (Group Lending)
    \item Diversificazione dei progetti: l'imprenditore anzichè investire in un solo progetto, se disponde di più idee, può diversificare l'investimento e investire in più progetti scorrelati tra loro per incrementare le possibilitá di successo
    \item Collaterizzazione: garanzie in caso di insolvenza date ai creditori
    \item Reputazione dell'imprenditore: se l'imprenditore ha uno storico importante di successi, la sua storia fa da garante per i suoi progetti futuri
    \item Convenants (clausole contrattuali): si dividono a loro volta in:
        \subitem Clausole del fare
        \subitem Clausole del non fare
        \subitem Clausole legate agli indici
\end{itemize}

\subsubsection{Micro credito (Group Lending)}

Non si offre credito ad un solo imprenditore ma si offre ad un gruppo di imprenditori legati tra di loro da un vincolo: ogni imprenditore risponde anche in caso di insolvenza di un'altro imprenditore dello stesso gruppo.

Questa tecnica di finanziamento ha successo soprattutto nei casi in cui c'è molta solidarietà tra gli imprenditori appartenenti al gruppo.

Le condizioni sono:
\begin{equation} \label{eq:conditionsMicro}
    \begin{gathered}
        A<\overline{A} \\
        0<a<1
    \end{gathered}    
\end{equation}
\begin{conditions*}
    A < \overline{A} &   Disponibilitá economica (\(A\)) molto inferiore rispetto alla disponibilitá economica del caso standard (\(\overline{A}\))  \\
    a   &   solidarietà tra gli impreditori appartenenti ad un Group Lending
\end{conditions*}

Il \textit{Vincolo di compatibilità degli incentivi} diventa:
\begin{equation}\label{eq:compatibilitaIncentiviMicro}
    \begin{gathered}
        p_H^2(R_B+a*R_B) \geq p_Hp_L(R_B+a*R_B) + B\\
        p_HR_B \geq \frac{B}{(1+a)\Delta p}
    \end{gathered}
\end{equation}
\begin{conditions*}
    p_HR_B &   Minimo ammontare \(R_B\) da attrivuire agli imprenditori per essere sicuri che scelga \(H\)
\end{conditions*}

Ogni imprenditore può \textit{Pledge}, ovvero ogni investitore ha un \textit{Pledgeble Income} di:

\begin{equation} \label{eq:pledgebleIncomeMicro}
    p_H(R-\frac{B}{(1-a)\Delta p})
\end{equation}
\begin{conditions*}
    \frac{B}{(1-a)\Delta p} > \frac{B}{\Delta p} &  Il denominatore è aumentato in questo caso di \(1-a\), valore correlato direttamente con l'intesa che c'è tra gli imprenditori del gruppo
\end{conditions*}

Il \textit{Pledgeble Income} è quindi direttamente proporzionale con l'intesa che c'è tra gli imprenditori del gruppo ed è sicuramente più alto rispetto al caso standard di credito ad un singolo imprenditore. In presenza di Micro Credito o Group Lending sará quindi più facile ottenere un finanziamento.

Il fatto che ci sia un interesse comune da incentivo agli imprenditori a fare la scelta realisticamente più redditizia, si tratta quindi di una soluzione al Moral Hazard

\subsubsection{Riepilogo Soluzioni Moral Hazard}

    \begin{enumerate}
        \item A chi interessa di più combattere il moral hazard? Al creditore, al debitore o al debitore potenziale?
        
        Il debitore potenziale è quello più coinvolto: il creditore suppone che in caso di concessione del finanziamento, si è giá tenuti conto del rischio di Moral Hazard, allo stesso modo il debitore durante la discussione delle clausole, se il progetto è giá stato finanziato allora il moral hazard è giá stato tenuto in conto. Il debitore potenziale è invece quello più interessato.
    \end{enumerate}

\section{Debt Overhang o Peso Eccessivo del Debito}

    Il Debt Overhang è un altro caso, come il Moral Hazard, in cui si verifica una diminuzione di disponibilitá a concedere credito. In presenza di un debito pregresso infatti i finanziatori saranno meno propensi a finanziare.

    I.E. La crisi del 2008 è un esempio concreto di Debt Overhang, dopo lo scoppio della bolla le banche concedevano pochissimo credito, molto meno anche del pre-crisi.

    Il \textit{Pledgeble Income} è uguale alla \ref{eq:pledgebleIncomeStandard}, il nuovo \textit{Vincolo di partecipazione} tiene però conto del debito pregresso \(D\):

    \begin{equation}
        p_H(R-\frac{B}{\Delta p} - (I-A) - D > 0)
    \end{equation}
    \begin{conditions*}
        D &  Debito pregresso che non va ai nuovi finanziatori ma ai finanziatori pregressi
    \end{conditions*}

    Isolando la \(A\):
    \begin{equation} \label{eq:vincoloPartecipazioneDebt}
        A > \overline{A}+D
    \end{equation}
   
    Il vincolo \ref{eq:vincoloPartecipazioneDebt} è quindi ancora più strinente rispetto a \ref{eq:vincoloPartecipazioneA}, l'imprenditore deve quindi investire tanti più soldi tanto più alto è il suo debito pregresso.

\subsection{Soluzioni al Debt Overhang}

    \begin{enumerate}
        \item Rinegoziare i debiti con i creditori: difficile da attuare se i creditori sono molti e hanno una piccola quota del debito ciascuno
        \item Ricapitalizzare il debito eccessivo
    \end{enumerate}

\subsubsection{Domande di riepilogo sul Debt Overhang}
È vero affermare che se i nuovi creditori che hanno giá acquistato il nostro debito allargano il \textit{Debt Overhang} il costo aggiunto peserà sugli azionisti dell'impresa molto di più rispetto al caso in cui non ci fosse questo debito pregresso?

Si, anche se siamo azionisti e non obbligazionisti non siamo al riparo da questo meccanismo, infatti il debito costerà molto di più.

\section{Asimmetrie informative, Selezione Avversa}
\label{sec:selezionaAvversa}

Fin'ora abbiamo parlato solo di Moral Hazard ma esiste anche la \textit{Selezione Avversa} ovvero le asimmetrie informative.

i.e. Nel 2008 il Libro, tasso di scambio di prestiti tra le banche, sfiorava il 20\% annuo, questo perchè le banche conoscevano solo lo stato personale e non lo stato delle altre, questa mancanza di fiducia rendeva estremamente costosi i prestiti anche interni.

Le conseguenze possono essere anche disastrose:
\begin{itemize}
    \item Market Breakdown: fallimento di mercato
    \item Over/Under Investimenti e Cross-subsidization: il secondo termine indica quando qualcuno paga troppo e sussidia, inconsapevolmente, chi paga troppo poco. Il caso più diffuso è quello delle tasse
    \item Razionamento del credito
\end{itemize}

\subsection{Razionamento Finanziario dovuto all'asimmetria informativa}

Si ipotizzi che non ci sia Moral Hazard, anche l'imprenditore ha quindi interesse nella redditivitá dell'impresa, esistono però due tipi di imprenditore:
\begin{itemize}
    \item tipo \(G\): paga \(R>0\) al tempo \(t=1\), può avere successo con probabilitá \(p\) e non avere successo con probabilitá \(1-p\)
    \item tipo \(B\): paga \(R>0\) al tempo \(t=1\), può avere successo con probabilitá \(q<p\) e non avere successo con probabilitá \(1-q\), \(B\) è meno bravo rispetto a \(G\), vale quindi la condizione \(1>p>q>0\)
\end{itemize}

Nel caso di informazione asimmetrica l'imprenditore conosce la propria abilitá, il finanziatore non la conosce quindi decide di stimare la probabilitá che l'imprenditore in questione sia di tipo \(G\), la probabilitá che l'imprenditore sia di tipo \(G\) quindi sia bravo è \(\alpha\). La qualitá media del progetto nell'ottica dell'investitore è rappresentata dalla seguente equazione:

\begin{equation} \label{eq:qualitaImprenditoreMedia}
    m=\alpha p + (1-\alpha)q
\end{equation}
\begin{conditions*}
    m & qualitá dell'imprenditore stimata con media probabilistica \\
    \alpha & probabilitá che l'imprenditore sia \textit{capace} \\
    p & probabilitá di \textit{successo} di un imprenditore \textit{capace} \\
    (1-\alpha) & probabilitá che l'imprenditore sia \textit{non capace} \\
    q & probabilitá di \textit{successo} di un imprenditore \textit{non capace}\\
\end{conditions*}

Il progetto di tipo \(G\) è efficiente, il progetto di tipo \(B\) non lo è:
\[
  \underbrace{pR-I}_{G} > 0 > \underbrace{qR-I}_{B}
\]

Questa situazione va verificata prima di firmare il contratto

\subsubsection{Ipotesi: Allocazione efficiente - informazione Simmetrica}

Gli investitori sanno quale progetto è di tipo \(G\) e quale progetto è di tipo \(B\) (\textit{first best}), si effettua quindi l'allocazione più efficiente possibile. In termini attesi l'imprenditore ottiene:
\begin{itemize}
    \item \(pR_b^G\) se il progetto è di tipo \(G\)
    \item \(pR_b^B\) se il progetto è di tipo \(B\)
\end{itemize}

I finanziatori sono in competizione tra loro, il profitto finale deve quindi essere nullo secondo la regola del mercato in concorrenza perfetta quindi:

\begin{itemize}
    \item \(p(R-R_b^G)=I\) se il progetto è di tipo \(G\)
    \item \(q(R-R_b^B)=I\) se il progetto è di tipo \(G\), cambia il reddito atteso rispetto al tipo \(G\)
\end{itemize}
\begin{conditions*}
    R_b^G & compenso dell'imprenditore \(G\)\\
    R_b^B & compenso dell'imprenditore \(B\)\\
    p(R-R_b^G) & reddito netto del finanziatore \\
    I & Costi
\end{conditions*}

La quota massima del reddito si ricava dalle equazioni sopra:

\begin{itemize}
    \item \(R_b^G = R - \frac{I}{p}\)
    \item \(R_b^B = R - \frac{I}{q}\)
\end{itemize}

Il compenso dell'investimento (cioè il tasso di interesse) sará inversamente proporzionale alla probabilitá di successo, nel caso in cui la probabilitá di successo sará maggiore (caso \(G\)) allora il tasso di interesse sará minore.

Dato \(qR<I\) l'unico modo per ottenere il finanziamento sarebbe quello di pagare il finanziatore con tutto il ricavato più un'ulteriore somma aggiuntiva. La conclusione è che nel caso di progetto \(B\) il finanziatore non investe.

La teoria dell'allocazione efficiente rimane comunque non realizzabile, l'informazione infatti non è simmetrica nella realtá, la conseguenza è che le banche firmano un solo tipo di contratto con tutti i tipi di imprenditore, sia nel caso in cui l'investimento sia un buon investimento (caso \(G\)), sia nel caso in cui non si accorgano che davanti si trovano un imprendtore non capace (caso \(B\)), il compenso dell'imprenditore sará quindi lo stesso e varrá generalmente \(R_b\):
\[
    m(R-R_b)=I  
\]
sapendo che:
\[
    m=\alpha p + (1-\alpha)q  
\]
il tasso di interesse viene implicitamente fissato dalla formula calcolata al tempo \(t=1\):
\begin{align*}
    m(R-R_b) &= I(1+i)\\
    i &= \frac{m(R-R_b)}{I}-1
\end{align*}

La conseguenza di questa informazione asimmetrica è duplice:
\begin{itemize}
    \item \(mR<1\) \(m\) è bassa quindi la media di successo è bassa, ci sono molti imprenditori di tipo \(B\), non avviene il finanziamento se il finanziatore scopre questo rendimento negativo in anticipo oppure se decide di investire c'è sottofinanziamento\footnote{Sottofinanziamento perchè i progetti buoni vengono sottofinanziati dando più spazio ai progetti di tipo \(B\), considerati un cattivo investimento}
    
        \[
            \left[ \alpha ^*p + (1-\alpha ^*)1 \right] R = I 
        \]
        \begin{conditions*}
            \alpha ^* & la minima frazione di progetti di tipo \(G\) che garantiscono un pareggio per gli investitori
        \end{conditions*}
        Il mercato fallisce, si verifica un \textit{market breakdown}, in questi casi è necessario aumentare l'informazione diminuendo le asimmetrie informative. I.E. durante la pandemia COVID19 lo stato è intervenuto facendo da garante per il credito richiesto dagli imprenditori alle banche. Il costo per lo stato è molto elevato perchè la Possibilitá di fallimento in questo periodo era molto alta, ma in questo modo il paramentro \(m\) non aveva più senso di esistere per le banche che non ne tennero conto concedendo credito in misura maggiore a seguito delle valutazioni standard. 

    \item \(R_b\) è tale che \(m(R-R_b=I)\), la banca finanzia sia gli imprenditori \(G\), sia gli imprenditori di tipo \(B\), le conseguenze sono molteplici tra cui:
        \begin{itemize}
            \item \(G\) pagato interessi più alti per colpa di B
                \begin{align*}
                    R-R_b &= I(1+r) \\
                    r &= \frac{R-R_b}{I}-1\\
                \end{align*}
                \(r\) è un tasso maggiore rispetto al caso ideale in cui gli imprenditori sono tutti di tipo \(G\)
                \[p(R-R_b)>I>q(R-R_b)\]
                quindi gli imprenditori \(G\) sussidiano gli imprenditori \(B\)
                \[
                    R_b = R - \frac{1}{m} < R_b^G = R - \frac{1}{p} 
                \]
            \item Il costo del capitale di \(G\) è troppo elevato quindi qualche imprenditore di tipo \(G\) non entra nel mercato
            \item Tanti progetti non efficienti di tipo \(B\) vengono finanziati comunque
        \end{itemize}
\end{itemize}

\subsubsection{Domande di riepilogo}
\begin{enumerate}
    \item Quanto più gli investitori pensano sia ridotta la quota di progetti efficienti nell'insieme dei progetti che richiedono finanziamenti tanto più è probabile che ci sia un market breakdown sul mercato del credito?
    
    È vero perchè la quota di progetti efficienti è la \(m\)

    \item Se il tasso di interesse sugli investimenti sicuri è positivo, allora è meno probabile vi sia un market breakdown?
    
    Falso
\end{enumerate}

\subsection{Market Timing}

La teoria dell'informazione asimmetrica ha delle conseguenze interessanti e inaspettate. Si supponga di aumentare la probabilitá di successo di \(G\) e \(B\) di un valore \(\tau\) noto a tutti:
\begin{align*}
    G &\rightarrow p + \tau\\
    B &\rightarrow q + \tau
\end{align*}

Negli occhi di un investitore, un aumento della probabilitá di successo sia per \(B\) che per \(G\) è equivalente a dire che gli imprenditori hanno un aumento della redditivitá su tutti quanti i progetti, non solo su una frazione di essi. Avere il \(\tau\) significa che il vincolo di partecipazione cambia anche se non si sa se si tratta di un progetto di tipo \(G\) o di tipo \(B\). La probabilitá di successo nell'ottica dell'investitore aumenta in media quindi gli investitori sono più ottimisti:
\begin{align*}
    \left[ \alpha(p+\tau) + (1-p)(1+\tau) \right](R-R_b) &= I\\
    \left[ \alpha(p+\tau) + (1-p)(1+\tau) \right]R &> I\\
    (m+\tau)R &> I\\
\end{align*}

quindi maggiore è \(\tau\) più c'è probabilitá che i progetti in generale vengano finanziati.

Il market timing prevede la cosiddetta teoria del ciclo economico \footnote{\url{https://it.wikipedia.org/wiki/Ciclo_economico}} secondo la quale saranno sempre presenti momenti di espansione alternati a momenti di market breakdown

\subsection{Certificazioni}

Le certificazioni sono dei segnali credivili eseguiti da un agente esterno che certifica a banche o altri enti esterni (o più in generale al mercato nella sua interezza) la qualitá della redditivitá delle imprese che ne richiedono la dimostrazione. Richiedere una certificazione ha un prezzo perchè eseguire le dovute valutazioni richiede un tempo e un lavoro non indifferenti. L'ente certificatore, per la sua reputazione, ha l'incentivo a dire la veritá.

Il costo di una certificazione \(c>0\) è pagato dalle imprese che hanno bisogno della certificazione e non da investitori o finanziatori dell'impresa, ma sono investitori e finanziatori che necessitano di queste informazioni. L'ente certificatore ha lo scopo di scoprire se l'impresa è di tipo \(B\) o di tipo \(G\). Se si considera il caso \(G\) definire \(R_b^G\) il massimo \textit{payoff} \(G\) che può ottenere in caso di certificazione:

\subsubsection{nuovo Vincolo di partecipazione}

\begin{equation}
    \begin{split}
        p(R-\widehat{R_b^G} )-c &= I\\
        R_b^G &= R - \frac{I+c}{p}
    \end{split}
\end{equation}

\begin{conditions*}
    \widehat{R_b^G} \neq R_b^G & infatti il compenso dell'imprenditore in caso di certificazione è diverso dal compenso dell'imprenditore in assenza della stessa \\
    c & Costo della certificazione, spesso non è trascurabile
\end{conditions*}

Se non si ottiene la certificazione si otterrá il risutlato:

\begin{align*}
    m(R-R_b)&=I\\
    R_b &= R - \frac{I}{m}
\end{align*}

É conveniente possedere una certificazione solo sotto le seguenti condizioni:

\begin{equation}
    \begin{split}
        R_b^G &> R_b\\
        R - \frac{I+c}{p} &> R - \frac{1}{m}
    \end{split}
\end{equation}

Quindi supponendo l'onestá  di tutte le parti in gioco, anche l'imprenditore ha convienienza ad essere certificato solo se valoglo le condizioni sopracitate.

Si sotituisce ora \(m\) con la sua definizione \ref{eq:qualitaImprenditoreMedia} e si ottiene:

\begin{equation}
    \frac{c}{1+c} < (1-\alpha)\frac{p-q}{p}
\end{equation}

\begin{conditions*}
    1-\alpha & Proporzione dei B (Bad)\\
    (1-\alpha)\frac{p-q}{p} & in presenza di pochi G (Good), la loro proporzione diventa trascurabile
\end{conditions*}

In questo modo si capisce come la certificazione venga richiesta tutte le volte che \(c\) non sia troppo elevato e che convenga certificare solo quando effettivamente l'asimmetria informativa sia troppo elevata. Se a destra il numero è molto alto allora significa che si è in presenza di un universo con pochi G e molti B, se la media intorno all'imprenditore che vuole essere finanziato è molto bassa allora diventa più importante distinguersi dalla massa e questa regola è una regola generale.

\subsection{Esempio Selezione avversa completo}

Un'impresa può investire in un progetto con le seguenti caratteristiche:
\begin{itemize}
    \item Costo \(I=50\), l'imprenditore non ha fondi propria
    \item Rendimento aleatorio:
        \begin{itemize}
            \item \(R=80\) in caso di successo 
            \item \(R=0\) in caso di insuccesso
        \end{itemize}
    \item La probabilitá di avere successo (\(p\)) dipende dal talento \(\Theta\) dell'imprenditore:
        \begin{itemize}
            \item se \(\Theta = T\) allora \(p=0.8\)
            \item se \(\Theta = NT\) allora \(p=0.4\)
        \end{itemize}
    \item L'imprenditore conosce il proprio \(\Theta\)
    \item La banca o più in generale il finanziatore non sa se \(\Theta=T\) o se \(\Theta=NT\) però sa che la probabilitá \(\pi\) che \(\Theta = T\) su questi progetti, storicamente è 0.5
    \item L'imprenditore che ottiene il credito è protetto da responsabilitá limitata
    \item Sia la banca che l'imprenditore sono neutrali al rischio
    \item Il costo opportunitá ad investire nel progetto per la banca è \(r=0\)
    \item Il contratto di credito è semplice:
        \begin{itemize}
            \item La banca presta \(I=50\) all'imprenditore
            \item L'imprenditore ripaga \(D\geq I\) (\(D\leq R\)) alla banca allo scadere del credito (quando il risultato \(R\) si realizza)
        \end{itemize}
\end{itemize}

\subsubsection{Caso First Best}

sis supponga che la banca conosca \(\Theta\) dell'imprenditore che chiede il finanziamento

Risultati:
\begin{itemize}
    \item Allocazione efficiente
    \item Solo gli imprenditori \(\Theta=T\) vengono finanziati
    \item La banca ha profitti attesi nulli, il tasso di interesse è pari al 25\%
    \item Imprenditore con \(\Theta = T\) guadagna l'intero \(NPV\) atteso del progetto
\end{itemize}

Si trova il \textit{Vincolo di partecipazione:}

\begin{align*}
    Reddito &= 50\\
    0.8\times D + 0.2\times 0 &= 50\\
    0.8\times D   &= 50\\
    D &= \frac{50}{0.8} = 62.5
\end{align*}

Se non si fa fallimento allora viene ripagato alla banca 62.5, il tasso di interesse che ne deriva è:

\[
    r = \frac{62.5}{50} - 1 = 0.25 
\]

L'interesse è quindi implicito nel rimborso, la banca in questo caso ottiene un tasso pari a 0 su un mercato alternativo, Dato che sugli investienti alternativi ottiene 0 allora anche sull'investimento in questione (globalmente) deve ottenere 0 (in media), per ottenere un atasso pari a 0 in media la banca deve ottenere un tasso pari a 0.25 in caso in cui le cose vadano per il verso giusto perchè quando le cose vanno male perde i soldi investiti.

Se invece si ha a che fare con un imprenditore di tipo B allora la probabilitá di successo è di 0.4:

\begin{align*}
    Reddito &= 50\\
    0.4\times D + 0.6\times 0 &= 50\\
    0.4\times D   &= 50\\
    D &= \frac{50}{0.4} = 125
\end{align*}

Si noti che \(125 > 80\), il rendimento è molto piú alto e questo non rappresenta un controsenso ma la giustificazione ad un maggior rischio

\subsubsection{Caso Informazione asimmetrica}

\begin{itemize}
    \item La banca non conosce il talento dell'imprenditore \(\Theta\)
    \item Assegna quindi una probabilitá pari al 50\%
\end{itemize}

\[
    E\left[R\right] = \underbrace{\frac{1}{2}\left[\cdots \right]}_{G} + \underbrace{\frac{1}{2}\left[\cdots \right]}_{B}
\]

\begin{conditions*}
    E\left[R\right] & Reddito medio
\end{conditions*}

Una volta calcolato il reddito medio occorre capire se sia almeno pari all'investimento di 50:

\begin{align*}
    E\left[R\right] &= \frac{1}{2}\left[0.8\times D + 0.2\times 0\right] + \frac{1}{2}\left[0.4\times D + 0.6\times 0\right] =\\
    &= 0.4\times D + 0.2\times D = \\
    &= 0.6\times D = 50 \Rightarrow D = \frac{50}{0.6} = 83.3
\end{align*}

Essendo maggiore di 80 nessuno viene finanziato, se qualcuno venisse finanziato le condizion isarebbero un rimborso di 83 anche in caso di successo, quindi anche se il progetto fruttasse 80. Nessuno accetterebbe di pagare di più di quello che riuscirebbe ad ottenere nel migliore dei casi.

Un modo alternativo di calcolare \(D\) è supporre che l'imprenditore abbia successo, ricavi quindi 80, poi restituisce il 100\% alla banca (caso limite), in questo caso infatti \[0.6\times 80 = 48\]

Essendo \(48<50\) significa che il finanziatore, anche nel migliore dei casi sarebbe in perdita e non rientrerebbe dei suoi costi, ipotesi non realistica

\subsection{Costo dello stato in caso di garanzia statale}

La \textit{probabilitá di fallimento totale} vale:
\begin{equation}
    P_{\text{fallimento}} = \frac{1}{2}(0.2) + \frac{1}{2}(0.6) = 0.4
\end{equation}

Essendo la probabilitá di fallimento totale pari a 0.4 significa che lo stato nel 40\% dei casi dovrá pagare 50, il costo per lo stato in questo caso è quindi di\[50 \times 0.4 = 20\]

20 è quindi il costo che deve prevedere lo stato nel fare garanzia di credito verso le banche per gli imprenditori


